\documentclass[a4paper,11pt]{article}

\usepackage[utf8]{inputenc}

\usepackage{graphicx}
\usepackage{caption}
\usepackage{subcaption}

\usepackage{pgfplots}
\pgfplotsset{compat=1.18} 

\usepackage{minted}

\begin{document}

\title{
  \textbf{Graphs and Shortest Path Java}
}
\author{Alexander Alvarez}
\date{HT P1 2024}

\maketitle

\section*{Introduction}

A graph data structure will be implemented and the shortest path 
between different pairs of nodes will be measured. The general 
behaviours of the graph data structure are described as well as how 
the operations were created and implemented for benchmarking.

\section*{Graph}

A graph data structure manages data using a cluster of nodes or 
$vertices$ where the connections between two vertices are called 
$edges$. Edges can be directed to be either one- or two-way. A 
collection of edges that connects two vertices that are not directly 
connected is called a $path$. 

There are several types of graphs but the type discussed in this report 
is called a connected undirected cyclic graph. A cyclic graph contains 
paths that form a cycle and an undirected graph has edges that are 
two-way. The graph being connected means that one can find a path 
from one vertex, that leads to any other vertex whithin the graph.

\subsection*{Map}

A CSV file is provided containing several connections between different 
pairs of Swedish cities with the distance in minutes. The graph is to be 
implemented as a map consisting of an array with vertices where each vertex 
is a different city. Each city has their own array of connections to other 
cities.

\subsection*{City and Connection}

The {\tt City} and {\tt Connection} class is declared before creating the
map. The {\tt City} class contains the instance variables:
\begin{itemize}
  \item {\tt name} which holds the name of the city
  \item {\tt next} which is a reference pointer to another city
  \item {\tt neighbours[]} which is an array of connections to other cities
  \item {\tt size} which is the size of the {\tt neighbours} array
\end{itemize}
The constructor applies the name to a city with a parameter variable and
declares the {\tt neighbours} array together with it's size. The use of 
the {\tt next} variable will become important later, but for now it's set 
to {\tt null} in the constructor.

\begin{minted}{java}
  public City(String name){
    this.name = name; // apply parameter name
    this.next = null; // next city is null
    this.size = 1; // start size 
    this.neighbours = new Connection[this.size]; // declare array
  }
\end{minted}

Over at the {\tt Connection} class we only have two instance variables 
and the constructor. The instance variables are {\tt city} which holds a
{\tt City} variable, and {\tt distance} which is an integer. The 
constructor has two parameters where one is a connected city which
is applied to the {\tt city} variable, and the other is the connection's 
distance which is applied to the {\tt distance} variable.

Back to the {\tt City} class we have an instance method {\tt connect}
which adds a city to the {\tt neighbours} array. The method checks if
the {\tt neighbours} array is empty and adds the connection if it is.
If the array is not empty, the method then compares to see if the 
connection already exists and discontinues if a match is found. If no 
match is found, then the method increments {\tt size} by 1 and a 
temporary array {\tt copy} is created with the new size and with the 
elements of {\tt neighbours}. The bigger {\tt copy} array is then set 
as the {\tt neighbours} array where the new connection finally can be 
added.

\begin{minted}{java}
  public void connect(City nxt, Integer dst){
    int i = 0; // start at index 0
    if(this.neighbours[i] != null){ // if not empty
      while(i < this.size) // compare loop
        if(this.neighbours[i++].city == nxt) return; // return if match
      Connection[] copy = new Connection[++this.size]; // increment size/make copy
      for(int j = 0; j < i; j++) copy[j] = this.neighbours[j]; // copy
      this.neighbours = copy; // set neighbours as copy
    }
    this.neighbours[i] = new Connection(nxt, dst); // add connection
  }
\end{minted}

The {\tt connect} method basically works like a dynamic stack that
only adds new entries. 

\subsection*{Creating the map}

The {\tt Map} class creates the map and includes the {\tt City} instance 
variable {\tt cities[]} which is the array containing all cities. The 
integer instance variable {\tt mod} is also included and will become 
important shortly.

The constructor declares the {\tt cities} array with size {\tt mod} where 
it then reads each row of the CSV file. Each row has 3 columns where 
connections between cities are set up as $from,to,distance$. The 
constructor runs the instance method {\tt lookup} on the cities of the 
first two columns. 

The {\tt lookup} method takes a city name from its string parameter 
together with the {\tt mod} variable and generates a hash index using a 
hash function. The code used for the hash function in this implementation 
is provided from the assignment document. The method checks if the 
{\tt cities} array has an empty entry at the generated index and adds a 
city if the entry is empty. If the entry is not empty, the method then
compares the name of the stored city to the city name from the parameter 
for which the stored city is returned if there's a match. If the names 
don't match then there's a collision which is handled using the {\tt next} 
variable from the {\tt City} class to go to the next city with the same 
index to check for a match. If no match is found by the time the tail node
is reached, then the new city is added to the {\tt next} pointer of the 
tail node.

\begin{minted}{java}
  public City lookup(String city){
    Integer indx = hash(city, mod); // generate hash index
    if(this.cities[indx] != null){ // if empty entry
      City current = this.cities[indx]; // current city
      City previous = null; // previous city
      while(current != null){ // until tail node is reached
        if(current.name.compareTo(city) == 0) return current; // if match
        previous = current; // set previous to current
        current = current.next; // move current to next city
      }
      previous.next = new City(city); // add to next of tail 
      return previous.next; // return city
    }
    this.cities[indx] = new City(city); // add to empty entry
    return this.cities[indx]; // return city
  }
\end{minted}

Back in the constructor, the cities returned from the {\tt lookup} method
are stored in two variables. The integer value of the distance from the 
third column is also retrieved and stored in a variable. The {\tt connect}
method is then run on both city variables where both cities has the same 
distance but each other as the {\tt connect} arguments. 

\begin{minted}{java}
      City one = lookup(col[0]); // get city "from"
      City two = lookup(col[1]); // get city "to"
      Integer dist = Integer.valueOf(col[2]); // get distance
      one.connect(two, dist); // connect from and to
      two.connect(one, dist); // connect to and from
\end{minted}

\section*{Shortest path}

In a new class {\tt Paths} a method {\tt shortest} was created to find the 
shortest path between two cities. I gave up after a couple of hours of 
trying to implement the naive version. This version has the instance 
variables:
\begin{itemize}
  \item {\tt path[]} which is used to keep track of visited cities
  \item {\tt sp} which is a pointer that points at the latest element in {\tt path}
  \item {\tt opt} which keeps track of the most optimal path
\end{itemize}
The constructor declares the {\tt path} array with size 54 which is the 
amount of different cities. The constructor also declares the variables
{\tt sp} and {\tt opt} by setting them to {\tt 0} and {\tt null} 
respectively. The {\tt path} array is going to work like a stack with the 
{\tt sp} variable as its stack pointer.

The {\tt shortest} method includes a {\tt Paths} variable {\tt p} as a 
parameter which I found was necessary to reset {\tt opt} when looping the 
method.

The method first compares the {\tt City} parameters {\tt from} and {\tt to}
and returns {\tt 0} of they match. It then checks if an optimal path exists 
where it then checks if the {\tt time} parameter is greater than {\tt opt} 
if it does exist. The method returns {\tt null} if the {\tt time} parameter 
is greater than {\tt opt}. The method also returns {\tt null} if {\tt from}
is equal to any city stored in {\tt path}. This is all to discontinue 
unnecessary searches or to prevent entering an endless cycle.

An integer {\tt shrt} is declared and the city {\tt from} is added to the
stack. For each of the {\tt from} city's connections ({\tt conn}), the 
{\tt time} variable (initially set to {\tt 0}) and the distance of 
{\tt conn} is added onto {\tt shrt}. The method is then run recursively
with the connected city {\tt conn.city} and time {\tt shrt} as arguments 
for the respective parameters {\tt from} and {\tt time}. 

The recursive method either returns {\tt null} or an integer value, but 
either is stored in a variable {\tt dst}. If {\tt dst} is not {\tt null} 
then it's value is added onto {\tt shrt} which the optimal path {\tt opt}
then is set to if either no optimal path exists or the distance of 
{\tt shrt} is less than {\tt opt}.

After going through each connection of the city {\tt from} it is then 
removed from the stack. The variable {\tt shrt} is set to as the most
optimal path found so far ({\tt opt}) and ultimately returned.

\begin{minted}{java}
  private static Integer shortest(City from, City to, Paths p, Integer time){
    if(from == to) return 0; // destination found
    if(p.opt != null && time > p.opt) return null; // if time is greater than opt
    for(int i = 0; i < p.sp; i++) // for each city in the stack
      if(p.path[i] == from) return null; // if from is in the stack
    Integer shrt = null; // declare shrt
    p.path[p.sp++] = from; // add from to stack
    for(int i = 0; i < from.size; i++){ // for each connection to from
      Connection conn = from.neighbours[i]; // get connection
      shrt = time + conn.distance; // time so far
      Integer dst = shortest(conn.city, to, p, shrt); // recursive
      if(dst != null){ // if path exists
        shrt += dst; // add dst time onto time so far
        if(p.opt == null || shrt < p.opt) p.opt = shrt; // if no opt path
      }                             // exists or if shrt is faster than opt
    }
    p.path[p.sp--] = null; // remove from of stack
    shrt = p.opt; // set shrt to fastet path so far
    return shrt; // return shrt
  }
\end{minted}

\subsection*{Benchmark}

Running the benchmark for the shortest paths between the cities provided
in the assignment document yields the results presented in table 1.

\begin{table}[ht]
  \begin{center}
  \begin{tabular}{l|c|c|c}
  \textbf{From} & \textbf{To} & \textbf{Distance} & \textbf{Runtime} \\
  \hline
  Malmö         & Göteborg    & $153 min$   & $0 ms$    \\
  Göteborg      & Stockholm   & $211 min$   & $0 ms$    \\
  Malmö         & Stockholm   & $273 min$   & $0 ms$    \\
  Stockholm     & Sundsvall   & $327 min$   & $1 ms$    \\
  Stockholm     & Umeå        & $517 min$   & $6 ms$    \\
  Göteborg      & Sundsvall   & $515 min$   & $2 ms$    \\
  Sundsvall     & Umeå        & $190 min$   & $180 ms$  \\
  Umeå          & Göteborg    & $705 min$   & $0 ms$    \\
  Göteborg      & Umeå        & $705 min$   & $4 ms$    \\
  Malmö         & Kiruna      & $1162 min$  & $52 ms$   \\
  \end{tabular}
  \caption{Shortest path between cities.}
  \label{tab:table1}
  \end{center}
\end{table}

The method managed to find paths that I, during benchmark observations,
believe are the shortest - or atleast close to the shortest - paths. 
I unfortunately have no data to compare these results to. However, this 
version of the {\tt shortest} method seem to include more efficient and/or
comprehesive ways of finding a path compared to what I could think of for
the naive version.

Running the benchmark for the shortest paths from Malmö to all other
cities and sorting by distance yields the results diplayed in figure 1.

\begin{figure}[ht!]
  \centering
  \begin{tikzpicture}
    \begin{axis}[
      grid=both, xmin=-2, xmax=55, ymin=-50, ymax=1200,
      major grid style={line width=.4pt,draw=gray!60},
      xlabel={City by index}, ylabel={Distance in $ min$},
      width=12cm, height=8cm]
      \addplot[mark=square, smooth, color=blue] table {allDst.dat};
    \end{axis}
  \end{tikzpicture}
  \caption{Distance from Malmö to cities increasingly further way.}
  \label{fig:plot1}
\end{figure}

\section*{Discussion}

The shortest path algorithm is an algorithm commonly used for the graph
data structure. The algorithm used in this report is called the 
depth-first-search (DFS) algorithm with the addition of the optimal path.
The DFS algorithm has the time complexity $O(V + E)$ where $V$ is the 
amount of vertices and $E$ is the amount of edges. The method uses this
algorithm when searching for a path. The inclusion of optimal path 
simply discontinues the search from a certain point when the path from 
that point becomes longer than the shortest path found yet.

\end{document}