\documentclass[a4paper,11pt]{article}

\usepackage[utf8]{inputenc}

\usepackage{graphicx}
\usepackage{caption}
\usepackage{subcaption}

\usepackage{pgfplots}
\pgfplotsset{compat=1.18} 

\usepackage{minted}

\begin{document}

\title{
  \textbf{Performance of the Add and Lookup Methods on a Binary Tree}
}
\author{Alexander Alvarez}
\date{HT P1 2024}

\maketitle

\section*{Introduction}

The time efficiency of the add and lookup methods from the binary tree 
data structure is to be measured on trees with varying sizes. The 
lookup and two different add methods will be benchmarked and their 
implementations will be discussed.

The general behaviours of the binary tree data structure are described 
as well as how the operations were created and implemented for 
benchmarking.

\section*{Method}

The benchmarks will be performed with different sizes, denoted {\tt n}. 
The time for one operation on each {\tt n} is the total runtime of 
looping the add and lookup methods divided by the amount of loops. Only 
the minimum time for each {\tt n} will be measured due to consistency.

The benchmark methods are run for a sufficient time before the actual 
benchmarks to enable the just-in-time compiler to make optimizations.

\section*{Binary tree}

Binary tree manages data using a tree-like structure. All nodes in the 
tree stems from the root which is the first node added. Every node has 
a left and right reference - like a branch splitting in two. A node 
where both branches are {\tt null} is a leaf. The tree is also sorted,
meaning smaller values are sorted to the left branches and vice versa.

A binary tree is initially created similarly to linked list but with a 
{\tt Node} class that has two reference variables, {\tt left} and 
{\tt right}, instead of one. The \emph{tree} node {\tt root} is then 
declared and set to {\tt null} in the constructor.

The binary tree utilizes several different operations for managing data. 
The operations of importance in this report are the {\tt add()} and 
{\tt lookup()} methods.

\subsection*{Recursive add}

Two versions of the add method will be benchmarked, one implemented
using recursion and the other using iteration.

The recursive method first checks if the tree is empty for which a node 
is added to root if it is. The private method is otherwise called which 
compares a given value with the data stored in the current node. 
Nothing happens if they match. If the value is lower, then the left 
branch of the current node is checked. A node is either added if it's 
empty or the method is called recursively on the left branch if it's 
not. The same is done on the right branch for when the value is higher.

\begin{minted}{java}
  private void add(Integer value){
    if(this.value == value) return; // return if match
    if(this.value > value) // if value is lower
      if(this.left != null) this.left.add(value); // recursive
      else this.left = new Node(value); // if empty left branch
    else // if value is higher
      if(this.right != null) this.right.add(value); // recursive
      else this.right = new Node(value); // if empty right branch
  }
\end{minted}

Two benchmarks are run where one has two trees constructed using a
ordered sequence of values and the other has one tree constructed using 
a random sequence. One of the orderely constructed trees has the median 
value of {\tt n} as its root while the other has {\tt 0}. The 
benchmarks yields the following results displayed in figure 1.

\begin{figure*}[ht!]
  \centering
  \begin{subfigure}[t]{0.47\textwidth}
    \begin{tikzpicture}
      \begin{loglogaxis}[
        grid=both, enlargelimits={abs=1.95},
        grid style={line width=.1pt, draw=gray!20},
        major grid style={line width=.2pt,draw=gray!60},
        xlabel={tree size (n)}, ylabel={time in $ ns$},
        width=5.8cm, height=5.8cm]
        \addplot[mark=square, color=blue] table {ordRecAdd1.dat};
        \addplot[mark=square, color=red] table {ordRecAdd2.dat};
      \end{loglogaxis}
    \end{tikzpicture}
    \caption{Ordered sequence. Blue: root = {\tt 0}}
    \label{fig:subplot1}
  \end{subfigure}
  ~
  \begin{subfigure}[t]{0.5\textwidth}
    \begin{tikzpicture}
      \begin{loglogaxis}[
        grid=both, enlargelimits={abs=1.95}, yminorticks=true,
        grid style={line width=.1pt, draw=gray!20},
        major grid style={line width=.2pt,draw=gray!60},
        xlabel={tree size (n)}, ylabel={time in $ ns$},
        width=5.8cm, height=5.8cm]
        \addplot[mark=square, color=blue] table {rndRecAdd.dat};
      \end{loglogaxis}
    \end{tikzpicture}
    \caption{Random sequence}
    \label{fig:subplot2}
  \end{subfigure}
  \caption{Benchmark of recursive add method}
  \label{fig:plot1}
\end{figure*}

\subsection*{Iterative add}

The iterative add method does a similar empty root check. Two variables 
are declared if root isn't empty. One points at the current node 
({\tt cur}) while the other points at the previous node from the
current ({\tt pre}). The tree is then traversed using {\tt cur} while
comparing the given value to the data stored in the {\tt cur} node. The 
left branch is taken if the value is lower and vice versa, whereas
nothing happens if there's a match. The new node can be added if
{\tt cur} reaches an empty branch for which it's added through the 
previous node to the corresponding branch.

\begin{minted}{java}
  if(root == null){ // check empty
    root = new Node(value); // add node
    return; // exit
  }
  Node cur = root; // current node
  Node pre = null; // previous node
  while(cur != null){ // while current isn't empty
    pre = cur; // set previous to current
    if(cur.value > value) cur = cur.left; // if lower value
    else if(cur.value < value) cur = cur.right; // if higher value
    else return; // exit if match
  }
  if(pre.value > value) pre.left = new Node(value); // if lower value
  else pre.right = new Node(value); // if higher value
\end{minted}

Two benchmarks are again run in the same way as the first. The 
benchmarks yields the following results displayed in figure 2.

\begin{figure*}[ht!]
  \centering
  \begin{subfigure}[t]{0.47\textwidth}
    \begin{tikzpicture}
      \begin{loglogaxis}[
        grid=both, enlargelimits={abs=1.95},
        grid style={line width=.1pt, draw=gray!20},
        major grid style={line width=.2pt,draw=gray!60},
        xlabel={tree size (n)}, ylabel={time in $ ns$},
        width=5.8cm, height=5.8cm]
        \addplot[mark=square, color=blue] table {ordIteAdd1.dat};
        \addplot[mark=square, color=red] table {ordIteAdd2.dat};
      \end{loglogaxis}
    \end{tikzpicture}
    \caption{Ordered sequence. Blue: root = {\tt 0}}
    \label{fig:subplot3}
  \end{subfigure}
  ~
  \begin{subfigure}[t]{0.5\textwidth}
    \begin{tikzpicture}
      \begin{loglogaxis}[
        grid=both, enlargelimits={abs=1.95},
        grid style={line width=.1pt, draw=gray!20},
        major grid style={line width=.2pt,draw=gray!60},
        xlabel={tree size (n)}, ylabel={time in $ ns$},
        width=5.8cm, height=5.8cm]
        \addplot[mark=square, color=blue] table {rndIteAdd.dat};
      \end{loglogaxis}
    \end{tikzpicture}
    \caption{Random sequence}
    \label{fig:subplot4}
  \end{subfigure}
  \caption{Benchmark for iterative add method}
  \label{fig:plot2}
\end{figure*}

\subsection*{Lookup}

The lookup operation in a sorted binary tree works in a similar manner 
to the binary search algorithm. The lookup method will therefore be 
compared to the binary search method from a previous assignment.

The lookup method is similar to the recursive add method. The method
checks if the provided key matches the value stored in the current node
and returns true if they match. Otherwise, the method will traverse the 
tree in the same way as in the recursive method and return false if the 
key isn't found.

\begin{minted}{java}
  private Boolean lookup(Integer key){
    if(this.value == key) return true; // return true if match
    if(this.value > key) // if key is lower
      if(this.left != null) return this.left.lookup(key); // recursive
      else return false; // if empty left branch
    else // if key is higher
      if(this.right != null) return this.right.lookup(key); // recursive
      else return false; // if empty right branch
  }
\end{minted}

The benchmark is run using one tree constructed with a random sequence.
The benchmark yields the following results displayed in figure 3. The 
graph for the lookup method's performance is displayed as blue while 
the binary search's is red.

\begin{figure}[ht!]
  \centering
  \begin{tikzpicture}
    \begin{loglogaxis}[
      grid=both,
      grid style={line width=.1pt, draw=gray!20},
      major grid style={line width=.2pt,draw=gray!60},
      xlabel={size (n)}, ylabel={time in $ ns$},
      width=5.9cm, height=5.9cm]
      \addplot[mark=square, color=blue] table {lookupTree.dat};
      \addplot[mark=square, color=red] table {searchBin.dat};
    \end{loglogaxis}
  \end{tikzpicture}
  \caption{Benchmark for lookup method}
  \label{fig:plot5}
\end{figure}

\section*{Discussion}

Both benchmarks (a) using the orderly constructed trees reaches the 
time complexity $ O(n)$ on both trees. This is due to the tree taking 
on a linear structure as every new value added is higher than the last
and is therefore added only to the rightmost branch. Adding the next
higher value means going through close to {\tt n} nodes. Having the
median value of {\tt n} be the root only adds shortcuts which is why
the red graph is slightly faster but has the same time complexity.

Both benchmarks (b) using randomly constructed trees displays the same 
time complexity $ O(log(n))$. This is mainly due to how the add method
creates and traverses the tree which is similar to the binary search
algorithm. Each node has a upper and lower half relative to their value.
One half is picked every time the method checks for a given value. This 
means that $ n$ is divided by $ 2^k$ where $ k$ is the number of checks 
until reaching a leaf, thus having the complexity $ O(log(n))$.

The iterative method performs slightly better than the recursive method
but the difference is negligible. The iterative method is however
simpler to understand if one finds recursiveness tricky. 

The graph for the lookup method is not as clean as the graph for binary 
search in figure 3. This may be due to how the tree is constructed 
which leads to the tree structure not being as homogeneous as how 
binary search partitions a sorted array. Some branch paths can 
therefore be longer than others which causes the unevenness. Both 
graphs does however display the time complexity $ O(log(n))$.

Although being more consistent, the binary search method uses arrays
which comes with it's pros and cons. Binary tree utilizes nodes which
is more efficient at memory usage as well as insertion and removal.
But using nodes also has it's drawbacks and one such drawback can be
observed in the graph which is the inconsistency.


\end{document}
