\documentclass[a4paper,11pt]{article}

\usepackage[utf8]{inputenc}

\usepackage{graphicx}
\usepackage{caption}
\usepackage{subcaption}

\usepackage{pgfplots}
\pgfplotsset{compat=1.18} 

\usepackage{minted}

\begin{document}

\title{
  \textbf{Search Efficiency in Sorted Arrays in Java}
}
\author{Alexander Alvarez}
\date{Fall 2024}

\maketitle

\section*{Introduction}

The time efficiency of different search operations will be measured in 
a sorted array with varying sizes. Two variants of the search operation
will be tested and compared to the search time on an unsorted array.

The general behaviour of these operations are to be described as well
as how they were created and implemented.

\section*{Method}

We are already familiar with the search algorithm from a previous
assignment. This report will expand on the search operation but
include the same test methods that was used in the previous
assignment. 

The values that the operations will search for (keys) will be randomly
generated and will not be sorted in any tests. The search methods will
be looped where the time will be taken in intervals right before and 
after the loop. The difference of the time intervals will be the time
to execute the search operation a {\tt loop} amount of times. The 
time difference divided by {\tt loop} will give the median execution 
time for one search operation. Only the minimal time values will be 
used for comparisons between operations due to consistency.

The search methods will be run for a sufficient time before the actual 
benchmark to enable the just-in-time compiler for time optimizations.

\subsection*{Unsorted search}

The unsorted search time will be the time that I will compare the
sorted search operations to. 

The unsorted search method creates two arrays ({\tt array} and 
{\tt keys}) with randomly generated elements in a random order. The 
pool of values that the random number generator (RNG) takes from is 
double the size of the {\tt array}, i.e. if the {\tt array} size is 
1000 then the pool size is 2000. The difference between the two arrays 
is that the {\tt array} array has the varying size {\tt n} while the 
{\tt keys} array will have the size {\tt loop}. 

The created arrays are then used in the {\tt for} loop between the time 
intervals. The element that is to be searched for ({\tt key}) is taken 
from the {\tt keys} array for each loop iteration. The unsorted search
operation is then used with the {\tt array} and the {\tt key} as 
arguments.

\begin{minted}{java}
  long t0 = System.nanoTime();
  for(int i = 0; i < loop; i++){
      int key = keys[i];
      unsorted_srch(array, key);
  }
  long t1 = System.nanoTime();
  return (t1 - t0);
\end{minted}

The unsorted search operation is a simple {\tt for} loop that compares
an element at a certain {\tt array} index to the {\tt key}. The 
{\tt for} loop stops when the same value is found, but will 
search through the whole {\tt array} if no same value was found.

\begin{minted}{java}
  public static boolean unsorted_srch(int[] array, int key){
    for(int index = 0; index < array.length; index++){
      if(array[index] == key) return true;
    }
    return false;
  }
\end{minted}

The {\tt loop} variable is set to $ 1,000$ and the {\tt array} sizes
start from $ 1,000$ and increases by doubling the size to $ 1,024,000$
({\tt n =} \{1000, 2000, ..., 512000, 1024000\}). Running the code with
these values yields the following graph:

\begin{figure}[h]
  \centering
  \begin{tikzpicture}
    \begin{axis}[
      xmin=0, xmax=1024000, ymin=0, ymax=150,
      xlabel=n, ylabel={time in $ ms$},
      width=8cm, height=5cm]
      \addplot[] table {searchUnsort.dat};
    \end{axis}
  \end{tikzpicture}
  \caption{Unsorted search time}
  \label{fig:plot1}
\end{figure}

The search time for an unsorted array with a million elements takes 
about $ 125 \mu s$.

\subsection*{Sorted search}

The sorted search method is almost identical to the unsorted search 
method with the differences being:

\begin{itemize}
  \item an added method to create a sorted array
  
  \item a change in the condition of the search operation
  
  \item an adjustment to the RNG pool
\end{itemize}

The elements of {\tt array} were randomly generated in a random order 
in the unsorted search method. This time a method is created that 
still randomly generates the elements but every next element has a 
greater value than the previous.

The sort array method has a {\tt for} loop that generates a 
random value from 0 to 10 and stores it in the {\tt nxt} variable in 
each iteration. The {\tt nxt} variable increases with the sum of the
random value plus the value of the {\tt nxt} variable from the 
previous iteration. This makes it so that the {\tt nxt} variable has a 
greater value in every next iteration while maintaining the RNG. 
However, duplicates may still occur if the RNG generates the number 0. 
To solve this we add 1 to the sum which makes the {\tt nxt} variable 
atleast 1 value greater in every next iteration.

\begin{minted}{java}
  public static int[] sort_array(int n, Random rnd){
    int[] array = new int[n];
    int nxt = 0;
    for(int i = 0; i < n; i++){
      nxt += rnd.nextInt(10) + 1;
      array[i] = nxt; 
    }
    return array;
  }
\end{minted}

The sorted search operation is almost the same as the
{\tt unsorted\_srch} operation from earlier. The sorted search 
operation is however optimized by one change in the condition of the 
{\tt if} statement. The {\tt if} statement will stop the search loop 
if the value of an element is the same or greater than the {\tt key}. 
This is because the {\tt key} wont be found when every next element is 
greater than the {\tt key} now that the {\tt array} is sorted.

The RNG pool is the same for both arrays in the unsorted search method
which is double the size of {\tt n}. The sort array method complicates
this in the sorted search method because it has an RNG pool of 10 times 
the size of {\tt n} - i.e. the maximum value is $ 10 \times n$ if the 
RNG always generates the greatest value which is highly unlikely. For
consistency I chose the RNG pool for the {\tt keys} to be 10 times the
size of {\tt n}.

Running the code for the sorted search method with the same values
as the unsorted search benchmark ({\tt loop =} $1000$ and 
{\tt n =} \{1000, 2000, ..., 512000, 1024000\}) yields the following 
graph:

\begin{figure}[h]
  \centering
  \begin{tikzpicture}
    \begin{axis}[
      xmin=0, xmax=1024000, ymin=0, ymax=150,
      xlabel=n, ylabel={time in $ ms$},
      width=8cm, height=5cm]
      \addplot[] table {searchSort.dat};
    \end{axis}
  \end{tikzpicture}
  \caption{Sorted search time}
  \label{fig:plot2}
\end{figure}

The search time for a sorted array with a million elements takes about 
$ 114 \mu s$.

\subsection*{Binary search}

The binary search method is very similar to the sorted search method
with the only difference being the usage of a whole different search 
algorithm.

The search algorithm in question is the binary search algorithm which
checks the middle element and systematically divides the {\tt array} 
in halves until the {\tt key} is found. This is done by having a two
variables ({\tt first} and {\tt last}) respectively pointing at the 
first and last index of the array. A {\tt while(true)} loop is then
made with four {\tt if} statements and a {\tt middle} variable that 
points to the element in the middle between the {\tt first} and 
{\tt last} index pointers. 

The search loop stops if the value of the {\tt middle} element is the 
same as the {\tt key}. If the {\tt middle} element is less than both
the {\tt key} and {\tt last} pointer - i.e. if the {\tt key} may be 
further to the {\tt last} pointer - then the {\tt first} pointer will 
point to where {\tt middle} is pointing and {\tt middle} will point to 
a new middle between {\tt first} and {\tt last}. Similarly, if 
{\tt middle} is greater than {\tt key} and {\tt first} - i.e. if the 
{\tt key} may be further to the {\tt first} pointer - then {\tt last} 
will point to the {\tt middle} element. When the {\tt middle} element 
is not the {\tt key}, the {\tt first} and {\tt last} pointers will
actually point to one index above and under {\tt middle} respectively 
to not include that {\tt middle} element that didn't have the 
{\tt key}. The search loop is stopped if no {\tt key} is found under
the conditions of the {\tt if} statements.

\begin{minted}{java}
  public static boolean binary_srch(int[] array, int key){
        int first = 0;
        int last = array.length - 1;
        while(true){
            int middle = (first + last)/2;
            if(array[middle] == key) return true;
            if(array[middle] < key && middle < last){
                first = middle + 1;
                continue;
            }
            else if(array[middle] > key && middle > first){
                last = middle - 1;
                continue;
            }
            else return false;
        }
    }
\end{minted}

Running the code for the binary search method with the same values
from the two previous benchmarks yields the following graph: 

\begin{figure}[h]
  \centering
  \begin{tikzpicture}
    \begin{axis}[
      xmin=0, xmax=1024000, ymin=20, ymax=100,
      xlabel=n, ylabel={time in $ ns$},
      width=8cm, height=5cm]
      \addplot[] table {searchBin.dat};
    \end{axis}
  \end{tikzpicture}
  \caption{Binary search time}
  \label{fig:plot3}
\end{figure}

The binary search time for a sorted array with a million elements takes 
about $ 93 ns$. The binary search algorithm divides the {\tt array} in
half every time it checks for the {\tt key}. This means that $ n$ is 
divided by $ 2^k$ where $ k$ is the number of checks until no more 
subarrays can be created. A function that roughly describes the 
execution time on an array of size $ n$: 
$$ O(log(n))$$

Looking at the graph, I estimated the time to search through an array
with 64 million elements to take no more than $ 300 ns$. My optimistic
guess was undeniably crushed by the output time I got from the test
which was more than double my guessed time at roughly $ 650 ns$.

\section*{Conclusion}

The time of the sorted search method compared to that of the unsorted 
search method shows a slight increase in performance. Decreasing the
RNG pool for the {\tt keys} array to the median max value of the last
element in the {\tt array} array improves the performance even more.
However, those results may be skewed in a way that is inaccurate which
is why they were not included to be compared.

The slight increase in performance of the sorted search method may
likely prove useful in cases where an array is already sorted. The 
same can not be said if the array is unsorted for which the increased
performance does not outweigh the cost of sorting the array.

The binary search method shows a significant increase in performance
compared to both the sorted and unsorted search methods. The 
performance of the binary search method increases exponentially as
the array size grows larger compared to the other two methods that has
has a linear performance. This makes the binary searh method a highly 
viable option for searching through arrays with huge sizes. The 
significantly increased performance overall likely outweighs the cost 
of sorting an unsorted array.

There is an overall increase in searching through data that is already
sorted.

\end{document}
