\documentclass[a4paper,11pt]{article}

\usepackage[utf8]{inputenc}

\usepackage{graphicx}
\usepackage{caption}
\usepackage{subcaption}

\usepackage{pgfplots}
\pgfplotsset{compat=1.18} 

\usepackage{minted}

\begin{document}

\title{
  \textbf{Static and Dynamic Stack Implementation for Reversed
  Polish Notation Calculator in Java}
}
\author{Alexander Alvarez}
\date{Fall 2024}

\maketitle

\section*{Introduction}

A calculator using reversed polish notation will be implemented
in Java using static and dynamic stack arrays. This report will
go over the creation of these stacks and how to implement them
to use reversed polish notation for calculation.

The general behaviour of these stacks are to be described as well
as how they were implemented.

\section*{Reversed polish notation}

The reversed polish notation (RPN) is a way of writing mathematical
expressions by writing the operand last. The use case of RPN for
this report is the fact that calculation becomes more simple 
by using a stack, and we don't need any parentheses. 

\section*{The stack}

The word "stack" in Java refers to an array that is used to store
elements. A stack consists of 

\begin{itemize}
  \item The stack array
  
  \item A stack pointer
  
  \item A size variable
  
  \item Two methods named {\tt push()} and {\tt pop()}
\end{itemize}

The stack pointer will point at an element at the top of the stack. 
Two methods {\tt push()} and {\tt pop()} are responsible for adding 
and removing elements in the stack and will also increment the stack 
pointer as they do. The stack array is used as the "stack" and its 
size is defined by the size variable. However, the size variable is 
either static or dynamic depending on wether the stack is static or 
dynamic.

The reason for using RPN for calculations with a stack is due to the
simplicity as previously mentioned. This is due to how the {\tt push()} 
and {\tt pop()} methods works on the stack. The {\tt push()} operation
will add an element to the top of the stack and will increment the
stack pointer to point at this element as the top element. The 
{\tt pop()} operation removes the top element and returns the previous
element as the top element in the stack by incrementing the stack
pointer again. This works well with RPN as we can write or push all
the elements we want to work with onto the stack and then pop them
out for the calculation. 

\subsection*{StaticStack}

The main difference between the static and dynamic stack is that
the array size of the static stack does not change but is defined
when the stack is created. The stack pointer is referred to as 
{\tt top} in the following code.

\begin{minted}{java}
  public StaticStack(int size){
    stack = new int[size];
    top = 0;
  }
\end{minted}

A fixed stack size enables something called "Stack Overflow" where 
pushing more elements than there is room for in the stack leads to 
an error. A solution to this can be to return a specific value when 
the stack pointer reaches the length of the stack, or an exception
can be thrown. I made it so that the {\tt push()} method will throw
{\tt ArrayIndexOutOfBoundsException} when the method detects that
there will be a stack overflow.

\begin{minted}{java}
  public void push(int val) throws ArrayInde....{
    if(top == (stack.length)) throw new Ar...("Stack Overflow");
    
    stack[top++] = val;
  }
\end{minted}

Another error will appear if we try to pop an element when the stack
is already empty. The {\tt pop()} method will try to return a value 
at index $ -1$ if the method is already empty. This is the same error
as the previous due to the method trying to access an element at an 
index that is out of bounds for the array size. I used the same 
solution to solve this error.

\begin{minted}{java}
  public int pop()throws ArrayInde...{
    if(top == 0) throw new ArrayI...("Stack is empty");
    
    return stack[--top];
  }
\end{minted}

This is basically it for the static stack. The static stack is stored
in the class {\tt StaticStack} to be used in our calculator later.

\subsection*{DynamicStack}

If the static stack has an array size that is constant and is defined
when the stack is created, then the dynamic stack is simply the 
opposite. 

We will have a default size so that the stack size does not become 
unnecessarily small when it changes.

\begin{minted}{java}
  public DynamicStack(){
    size = 4;
    stack = new int[size];
    top = 0;
  }
\end{minted}

A dynamic stack size does not generate an error when we try to push
more elements than there is room for in the stack. The stack will
instead increase in size to fit more elements. This is done by 
creating a new, bigger stack and copying over all the elements from
the previous stack to the new stack when the stack pointer reaches 
the size limit instead of throwing an exception. 

The size variable will be updated to the new size which is chose to 
be double the size of the stack.

\begin{minted}{java}
  public void push(int value){
    if(top == (size)){
      int copy[] = new int[size*2];
      for(int i = 0; i < size; i++) copy[i] = stack[i];
      stack = copy;
      size = stack.length;
    }

    stack[top++] = value;
  }
\end{minted}

The {\tt pop()} method will generate the same "out of bounce" error
for the dynamic stack if we try to pop an element in an already empty 
stack.

The {\tt pop()} method in the dynamic stack serves another function 
which is to decrease the stack size if the bigger stack size is not
being utilized. This is done in a similar way to how the {\tt push()}
method increases the stack size. The {\tt pop()} method creates a new,
smaller stack and copies over all the elements from the previous stack 
to the new stack when the stack pointer points to a certain index.

The size variable will be updated to the new size which is chose to 
be half the size of the stack. We leave some margins for the method
to work with so that the size does not immediately decrease when the
stack pointer points at half the stack size. This is done by simply
making less than half the stack size the point of execution. We
also don't want to decrease the stack size beyond the default size.

\begin{minted}{java}
  public int pop()throws ArrayInde...{
    if(top == 0) throw new ArrayI...("Stack is empty");
    
    if(((top + 1) < (size / 2)) && (size != 4)){
      int copy[] = new int[size/2];
      for(int i = 0; i < copy.length; i++) copy[i] = stack[i];
      stack = copy;
      size = stack.length;
    }

    return stack[--top];
  }
\end{minted}

The dynamic stack is very similar to the static stack with the
difference being how they manage their sizes. The dynamic stack 
is stored in the class {\tt DynamicStack} to later be used in our 
calculator. 

\section*{The Calculator}

The calculator itself is not too difficult to make. We create our 
stack and the reader for the terminal inputs. The integer inputs is
pushed onto the stack until the reader detects an operation. If an
operation is inserted, the the program will switch to the 
corresponding case to perform the calculation.

The respective operation cases are almost identical to eachother,
difference being the operand in the respective {\tt for} loops.
The calculation stores the top element of the stack on a variable and
performs the corresponding operation on this variable with each of
the remaining elements in the stack. The result is then pushed onto
the stack to be printed out or to be used in further operations.

The elements are in the wrong order in the case for the subraction 
operation. The problem is solved by multiplying the result with $ -1$
before pushing it to the stack.

\begin{minted}{java}
  while(run){
    String input = br.readLine();
    switch (input){
      case "+":
        int sum = stack.pop();
        for(int i = stack.top; i > 0; i--){
          sum += stack.pop();
        }
        stack.push(sum);
        break;

      .
      .
      .

      case "":
        run = false;
        break;
      
      default:
        Integer nr = Integer.parseInt(input);
        stack.push(nr);
        break;
    }
  }
\end{minted}

Running the code and inserting the given input of 
"$ 4 2 3 * 4 + 4 * + 2 -$" we get $ 110$ as the output result. The
result is the same for both the static and dynamic stack, provided
that the static stack is of atleast length 3. 

\section*{Conclusion}

Both the static and dynamic stack was successfully implement using
RPN. The output result is the same solution as my solution solved
on paper, so I take it as the calculator being successful. 

My biggest issue was implementing generic and abstract classes
for the stacks, but i skipped this since it was not mandatory 
according to the assignment description.

\end{document}
